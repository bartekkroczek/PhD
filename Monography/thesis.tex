% Options for packages loaded elsewhere
\PassOptionsToPackage{unicode}{hyperref}
\PassOptionsToPackage{hyphens}{url}
%
\documentclass[
  man,draftall]{apa6}
\usepackage{amsmath,amssymb}
\usepackage{lmodern}
\usepackage{iftex}
\ifPDFTeX
  \usepackage[T1]{fontenc}
  \usepackage[utf8]{inputenc}
  \usepackage{textcomp} % provide euro and other symbols
\else % if luatex or xetex
  \usepackage{unicode-math}
  \defaultfontfeatures{Scale=MatchLowercase}
  \defaultfontfeatures[\rmfamily]{Ligatures=TeX,Scale=1}
\fi
% Use upquote if available, for straight quotes in verbatim environments
\IfFileExists{upquote.sty}{\usepackage{upquote}}{}
\IfFileExists{microtype.sty}{% use microtype if available
  \usepackage[]{microtype}
  \UseMicrotypeSet[protrusion]{basicmath} % disable protrusion for tt fonts
}{}
\makeatletter
\@ifundefined{KOMAClassName}{% if non-KOMA class
  \IfFileExists{parskip.sty}{%
    \usepackage{parskip}
  }{% else
    \setlength{\parindent}{0pt}
    \setlength{\parskip}{6pt plus 2pt minus 1pt}}
}{% if KOMA class
  \KOMAoptions{parskip=half}}
\makeatother
\usepackage{xcolor}
\usepackage{graphicx}
\makeatletter
\def\maxwidth{\ifdim\Gin@nat@width>\linewidth\linewidth\else\Gin@nat@width\fi}
\def\maxheight{\ifdim\Gin@nat@height>\textheight\textheight\else\Gin@nat@height\fi}
\makeatother
% Scale images if necessary, so that they will not overflow the page
% margins by default, and it is still possible to overwrite the defaults
% using explicit options in \includegraphics[width, height, ...]{}
\setkeys{Gin}{width=\maxwidth,height=\maxheight,keepaspectratio}
% Set default figure placement to htbp
\makeatletter
\def\fps@figure{htbp}
\makeatother
\setlength{\emergencystretch}{3em} % prevent overfull lines
\providecommand{\tightlist}{%
  \setlength{\itemsep}{0pt}\setlength{\parskip}{0pt}}
\setcounter{secnumdepth}{-\maxdimen} % remove section numbering
% Make \paragraph and \subparagraph free-standing
\ifx\paragraph\undefined\else
  \let\oldparagraph\paragraph
  \renewcommand{\paragraph}[1]{\oldparagraph{#1}\mbox{}}
\fi
\ifx\subparagraph\undefined\else
  \let\oldsubparagraph\subparagraph
  \renewcommand{\subparagraph}[1]{\oldsubparagraph{#1}\mbox{}}
\fi
\newlength{\cslhangindent}
\setlength{\cslhangindent}{1.5em}
\newlength{\csllabelwidth}
\setlength{\csllabelwidth}{3em}
\newlength{\cslentryspacingunit} % times entry-spacing
\setlength{\cslentryspacingunit}{\parskip}
\newenvironment{CSLReferences}[2] % #1 hanging-ident, #2 entry spacing
 {% don't indent paragraphs
  \setlength{\parindent}{0pt}
  % turn on hanging indent if param 1 is 1
  \ifodd #1
  \let\oldpar\par
  \def\par{\hangindent=\cslhangindent\oldpar}
  \fi
  % set entry spacing
  \setlength{\parskip}{#2\cslentryspacingunit}
 }%
 {}
\usepackage{calc}
\newcommand{\CSLBlock}[1]{#1\hfill\break}
\newcommand{\CSLLeftMargin}[1]{\parbox[t]{\csllabelwidth}{#1}}
\newcommand{\CSLRightInline}[1]{\parbox[t]{\linewidth - \csllabelwidth}{#1}\break}
\newcommand{\CSLIndent}[1]{\hspace{\cslhangindent}#1}
\ifLuaTeX
\usepackage[bidi=basic]{babel}
\else
\usepackage[bidi=default]{babel}
\fi
\babelprovide[main,import]{english}
% get rid of language-specific shorthands (see #6817):
\let\LanguageShortHands\languageshorthands
\def\languageshorthands#1{}
% Manuscript styling
\usepackage{upgreek}
\captionsetup{font=singlespacing,justification=justified}

% Table formatting
\usepackage{longtable}
\usepackage{lscape}
% \usepackage[counterclockwise]{rotating}   % Landscape page setup for large tables
\usepackage{multirow}		% Table styling
\usepackage{tabularx}		% Control Column width
\usepackage[flushleft]{threeparttable}	% Allows for three part tables with a specified notes section
\usepackage{threeparttablex}            % Lets threeparttable work with longtable

% Create new environments so endfloat can handle them
% \newenvironment{ltable}
%   {\begin{landscape}\centering\begin{threeparttable}}
%   {\end{threeparttable}\end{landscape}}
\newenvironment{lltable}{\begin{landscape}\centering\begin{ThreePartTable}}{\end{ThreePartTable}\end{landscape}}

% Enables adjusting longtable caption width to table width
% Solution found at http://golatex.de/longtable-mit-caption-so-breit-wie-die-tabelle-t15767.html
\makeatletter
\newcommand\LastLTentrywidth{1em}
\newlength\longtablewidth
\setlength{\longtablewidth}{1in}
\newcommand{\getlongtablewidth}{\begingroup \ifcsname LT@\roman{LT@tables}\endcsname \global\longtablewidth=0pt \renewcommand{\LT@entry}[2]{\global\advance\longtablewidth by ##2\relax\gdef\LastLTentrywidth{##2}}\@nameuse{LT@\roman{LT@tables}} \fi \endgroup}

% \setlength{\parindent}{0.5in}
% \setlength{\parskip}{0pt plus 0pt minus 0pt}

% Overwrite redefinition of paragraph and subparagraph by the default LaTeX template
% See https://github.com/crsh/papaja/issues/292
\makeatletter
\renewcommand{\paragraph}{\@startsection{paragraph}{4}{\parindent}%
  {0\baselineskip \@plus 0.2ex \@minus 0.2ex}%
  {-1em}%
  {\normalfont\normalsize\bfseries\itshape\typesectitle}}

\renewcommand{\subparagraph}[1]{\@startsection{subparagraph}{5}{1em}%
  {0\baselineskip \@plus 0.2ex \@minus 0.2ex}%
  {-\z@\relax}%
  {\normalfont\normalsize\itshape\hspace{\parindent}{#1}\textit{\addperi}}{\relax}}
\makeatother

% \usepackage{etoolbox}
\makeatletter
\patchcmd{\HyOrg@maketitle}
  {\section{\normalfont\normalsize\abstractname}}
  {\section*{\normalfont\normalsize\abstractname}}
  {}{\typeout{Failed to patch abstract.}}
\patchcmd{\HyOrg@maketitle}
  {\section{\protect\normalfont{\@title}}}
  {\section*{\protect\normalfont{\@title}}}
  {}{\typeout{Failed to patch title.}}
\makeatother

\usepackage{xpatch}
\makeatletter
\xapptocmd\appendix
  {\xapptocmd\section
    {\addcontentsline{toc}{section}{\appendixname\ifoneappendix\else~\theappendix\fi\\: #1}}
    {}{\InnerPatchFailed}%
  }
{}{\PatchFailed}
\keywords{attentional vigilance, antisaccade task, processing speed\newline\indent Word count: X}
\DeclareDelayedFloatFlavor{ThreePartTable}{table}
\DeclareDelayedFloatFlavor{lltable}{table}
\DeclareDelayedFloatFlavor*{longtable}{table}
\makeatletter
\renewcommand{\efloat@iwrite}[1]{\immediate\expandafter\protected@write\csname efloat@post#1\endcsname{}}
\makeatother
\usepackage{lineno}

\linenumbers
\usepackage{csquotes}
\ifLuaTeX
  \usepackage{selnolig}  % disable illegal ligatures
\fi
\IfFileExists{bookmark.sty}{\usepackage{bookmark}}{\usepackage{hyperref}}
\IfFileExists{xurl.sty}{\usepackage{xurl}}{} % add URL line breaks if available
\urlstyle{same} % disable monospaced font for URLs
\hypersetup{
  pdftitle={Does varying cue-stimulus interval affect the sensory discrimination performance in the antisaccade task?},
  pdfauthor={Bartłomiej Kroczek1 \& Adam Chuderski2},
  pdflang={en-EN},
  pdfkeywords={attentional vigilance, antisaccade task, processing speed},
  hidelinks,
  pdfcreator={LaTeX via pandoc}}

\title{Does varying cue-stimulus interval affect the sensory discrimination performance in the antisaccade task?}
\author{Bartłomiej Kroczek\textsuperscript{1} \& Adam Chuderski\textsuperscript{2}}
\date{}


\shorttitle{Attention variance over time}

\authornote{

This manuscript is the doctoral thesis of Bartłomiej Kroczek, prepared under the supervision of professor Adam Chuderski, implemented as part of the CogNes 19 doctoral program.

All code and data used in performing this research are publicly available at \texttt{https://github.com/bartekkroczek/PhD}

The authors made the following contributions. Bartłomiej Kroczek: Conceptualization, Formal Analysis, Investigation, Methodology, Software, Visualization, Validation, Writing - Original Draft Preparation, Writing - Review \& Editing; Adam Chuderski: Conceptualization, Methodology, Writing - Review \& Editing, Supervision.

Correspondence concerning this article should be addressed to Bartłomiej Kroczek. E-mail: \href{mailto:bartek.kroczek@doctoral.uj.edu.pl}{\nolinkurl{bartek.kroczek@doctoral.uj.edu.pl}}

}

\affiliation{\vspace{0.5cm}\textsuperscript{1} Institute of Psychology, Jagiellonian University, Krakow, Poland\\\textsuperscript{2} Department of Cognitive Science, Jagiellonian University, Krakow, Poland}

\abstract{%
One or two sentences providing a \textbf{basic introduction} to the field, comprehensible to a scientist in any discipline.

Two to three sentences of \textbf{more detailed background}, comprehensible to scientists in related disciplines.

One sentence clearly stating the \textbf{general problem} being addressed by this particular study.

One sentence summarizing the main result (with the words ``\textbf{here we show}'' or their equivalent).

Two or three sentences explaining what the \textbf{main result} reveals in direct comparison to what was thought to be the case previously, or how the main result adds to previous knowledge.

One or two sentences to put the results into a more \textbf{general context}.

Two or three sentences to provide a \textbf{broader perspective}, readily comprehensible to a scientist in any discipline.
}



\begin{document}
\maketitle

\hypertarget{rozdziaux142-1}{%
\section{Rozdział 1}\label{rozdziaux142-1}}

Tu mam najtrudniej, bo złapałem się na luce w rozumieniu. Jaki - z naszej perspektywy - jest związek między attentional vigilance a cognitive control? Czemu torturując zadanie antysakadowe które ma być miarą kontroli wysnuwamy wnioski o uwadze? To jest tak, że cognitive control to jakiś puzel, który służy do budowania teorii jak działa uwaga (i pewnie nie tylko)?

Pomijając powyższe, wyobrażam sobie takie fakty:

\begin{itemize}
\tightlist
\item
  Argumenty za tym, że uwaga jest zmienna w czasie (bo mamy np. mrugnięcie uwagowe, saccadic supression, oscylacyjne teorie wszystkiego itp).
\item
  Wykazanie, że zadanie antysakadowe jest powszechnie poważane jako dobra miara kontroli
\item
  Brakujący mi klej między kontrolą a uwagą
\item
  Argumenty literaturowe za istnieniem trendu i oscylacji (behawioralnych) w varying cue-stimuli interval
\item
  Pointa, że to się klei, że skoro uwaga ma być niestabilna w czasie, to i zadanie je mierzące powinno być niestabilne w czasie i to właśnie pokazuje literatura. (Znów, brak kleju)
\end{itemize}

\hypertarget{rozdziaux142-2}{%
\section{Rozdział 2}\label{rozdziaux142-2}}

Zarysowanie problemu, a mianowicie, że chcemy na własne oczy sprawdzić co to się dzieje w tym varying cue-stimuli interval.
Tutaj nie wiem na ile warto/należy spoilerować zakończenie.

\hypertarget{chapter-iii.what-our-data-shows}{%
\section{\texorpdfstring{Chapter III.\newline What our data shows?}{Chapter III.What our data shows?}}\label{chapter-iii.what-our-data-shows}}

To examine the role of CSI in stimulus discrimination, we used the antisaccade
task with CSI varied in millisecond steps. On each trial, participants were shown for 250 ms
either left or right arrow as a target, and their task was to press the appropriate
key. The arrows were randomly displayed on either the left or right side of the
screen. The stimulus was accompanied by a red dot briefly flashing on the
opposite side of the screen, which should be ignored. The fixation point and
the red dot/stimulus presentation were separated by a blank screen shown for
CSI ranging from 400 to 900ms. We conducted two experiments, one with lower
resolution (less trials per participant) but a larger sample (N=150, CSI sampling
frequency 60Hz) and the other with higher resolution but a smaller sample
(N=40, CSI sampling frequency 120Hz). Both data sets were analyzed in the
same way. For each participant, the mean accuracy was calculated for each CSI
timepoint. These mean accuracy values were plotted as a function of increasing
CSI, with a single curve created for each person.

Co o varying cue-stimuli interval mówią nam nasze eksperymenty

\hypertarget{sekcja-3.1}{%
\subsection{Sekcja 3.1}\label{sekcja-3.1}}

Dane N=150, niskie próbkowanie CSI 200-1000 ms. Już zebrane i przeanalizowane.
Wniosek - jest trend, ale malutki, dużo mniejszy niż w literaturze (i hope so)

\hypertarget{methods}{%
\subsubsection{Methods}\label{methods}}

We report how we determined our sample size, all data exclusions (if any), all manipulations, and all measures in the study.

\hypertarget{sekcja-3.2}{%
\subsection{Sekcja 3.2}\label{sekcja-3.2}}

Dane N=40 wysokie próbkowanie, CSI 400-900 ms. Już zebrane i przeanalizowane. Wniosek, brak trendu, zarówno w analizie zbiorczej dla każdej z częstotliwości (to od Tomasza) jak i różnic indywidualnych (kończę robić)

\hypertarget{sekcja-3.3}{%
\subsection{Sekcja 3.3}\label{sekcja-3.3}}

Dane N=Przyzwoicie\_Ale\_Nie\_Za\_Duzo niskie próbkowanie, CSI 500(?)-3000(?)ms Nie zebrane.
Wniosek, daj Bóg, że trend znika.

\hypertarget{rozdziaux142-4}{%
\section{Rozdział 4}\label{rozdziaux142-4}}

Dlaczego nasze wyniki nie pokrywają się z literaturą?
Sekcja 4.1 Trend
Bo efekt wisiał na punkcie w 200 ms czyli na mrugnięciu uwagowym.
Sekcja 4.2 Oscylacje
Wnioski i symulacje z artykułu metodologicznego o niepoprawnym preprocessigu. W końcu jest to już, marnie bo marnie, ale napisane.

Konkluzja: Nasze dane pokazują, że cue-stimuli interval nie oscyluje i ma bardzo mały/nie ma efektu trendu.

\hypertarget{rozdziaux142-5}{%
\section{Rozdział 5}\label{rozdziaux142-5}}

Nasze obserwacje nie zgadzają się z obecnym stanem rzeczy. Co gorsza, twierdzimy że wiemy dlaczego. Nasi biografowie nazwą to zjawisko Kroczek\&Chuderski Tension w analogii do Hubble Tension - największego kryzysu współczesnej kosmologii. W tym momencie czytelnik ma już mokre poty i zjadł wszystkie paznokcie. Gdy myśli, że jego nerwy nie mogą być już bardziej napięte to walimy go w łeb obuchem stawiając pytanie, czy w takim razie uwaga/kontrola jest jednak stała w czasie, czy zadanie antysakadowe nie jest dobrą miarą kontroli. Czytelnik mdleje.

\hypertarget{rozdziaux142-6}{%
\section{Rozdział 6}\label{rozdziaux142-6}}

W oparciu o Roudera stawiamy tezę, że zadanie antysakadowe nie mierzy kontroli i przedstawiamy kolejny eksperyment który temu dowodzi.
Opisujemy eksperyment i pokazujemy, że zadanie antysakadowe nie tyle jest marną miarą kontroli, co jest zadaniem na processing speed.

\hypertarget{rozdziaux142-7}{%
\section{Rozdział 7}\label{rozdziaux142-7}}

Konkluzja pracy, w jakich obszarach (teorie uwagi) nasze wnioski mogą namieszać. Future directions. Tłum wiwatuje w ekstazie. Kurtyna.

\hypertarget{references}{%
\section{References}\label{references}}

\hypertarget{refs}{}
\begin{CSLReferences}{0}{0}
\end{CSLReferences}


\end{document}
